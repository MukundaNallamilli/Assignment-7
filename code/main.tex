\documentclass{beamer}

\usepackage{amssymb}
\usepackage{amsfonts}
\usepackage{amsmath}
\usepackage{amsthm}
\usepackage{setspace}
\usepackage{longtable}
\usepackage{graphicx}
\usepackage{mathtools}
\usepackage{color}
\usepackage{array}
\usepackage{calc} 
\usepackage{bm}
\usepackage{caption}
\usepackage{float}

\usetheme{CambridgeUS}
\useoutertheme{infolines}
%numbering
\setbeamercolor{background canvas}{bg=white}
\setbeamersize{text margin left=1cm,text margin right=1cm}

\title[AI1110  Assignment-7]{ASSIGNMENT-7}
\subtitle{AI1110}
\author[]{MUKUNDA REDDY \\ AI21BTECH11021}
\date{}

\begin{document}
  \begin{frame}
      \titlepage
  \end{frame}
  
  \begin{frame}{Outline}
      \tableofcontents
  \end{frame}
  
  \section{Question}
  \begin{frame}{Example 6-43}
  Suppose $\textbf{x}$ and $\textbf{y}$ are independent binomial random variables
  with parameters $(m, p)$ and $(n, p)$ respectively. Then $\textbf{x} + \textbf{y}$ is also binomial with parameter $(m + n, p)$, so that \\
  $$ P\{x=x|\textbf{x}+\textbf{y}=x+y\} = \frac{P\{x=x\}P\{y=y\}}{P\{x+y=x+y\}} = \frac{(\frac{m}{n})(\frac{n}{y})}{(\frac{m+n}{x+y})}  $$ \\
  Thus the conditional distribution of $\textbf{x}$ given $\textbf{x} + \textbf{y}$ is
  hypergeometric. Show that the converse of this result which states that if $x$ and $y$ are non negative independent random
  variables such that $P\{x=0\}>0,P\{y=0\}>0$ and the conditional
  \end{frame}
  
  \begin{frame}{Example 6-43}
  Show that the converse of this result which states that if
   $x$ and $y$ are non negative independent random
   variables such that $P\{x=0\}>0$,$P\{ y=0\}>0$ and the conditional
   distribution of x given x + y is hyper geometric,then x and y are binomial random variables ?
  \end{frame}
  
  \section{Solution}
  \begin{frame}{Solution}
      We know that
      $$\frac{P \{ x= x \} }{{m \choose k}} \frac{P \{ y=y \} }{{n \choose y} }= \frac{P\{x+y=x+y\}}{{{m+n}\choose{x+y}}}$$
      let 
      $$\frac{P \{ x= x \} }{{m \choose k}} = f(x),\frac{P \{ y=y \} }{{n \choose y}} = g(y),\frac{P\{x+y=x+y\}}{ {{m+n} \choose {x+y}}} = h(x+y)$$
      \begin{center}
      Then h(x+y) = f(x) \times g(y)
      \end{center}
  \end{frame}
  
  \begin{frame}{Solution}
      and Hence
      \begin{center}
          $h(1) = f(1)g(0) = f(0)g(1)$ \\ \vspace{6pt}
          $h(2) = f(2)g(0) =f(1)g(1) = f(0)g(2)$  \\ \vspace{6pt}
         Generalizing $h(k) = f(k)g(0) = f(k-1)g(1)=...$ \\
      \end{center}
      Thus 
      $$f(k)=f(k-1)\frac{g(1)}{g(0)} = f(0)\left( \frac{g(1)}{g(0)} \right)^{k} $$
  \end{frame}
  
  \begin{frame}{Solution}
      or
      $$P\{x=k\} = {m \choose k}P{x=0}a^k \:\:  k=0,1,2....$$
     where    $a=\frac{g(1)}{g(0)} >0$.But $\sum P\{x=k\}=1$
     gives $P\{x=0\}(1+a)^m = 1$ or $P\{x=0\} = q^m$ where
     $q=\frac{1}{1+a} <1$.Hence $a=\frac{p}{q}$ where $p=1-q 
     >0$.We know that
     $$ P\{x=k\} = {m \choose k}p^k q^{m-k} , k=0,1,2....$$
  \end{frame}
  
  \begin{frame}{Solution}
      Similarly we can also show 
      $$ P\{y=r\}  = {n \choose r}p^r q^n-r, r=0,1,2...n$$ 
      Hence the proof is completed.
  \end{frame}
  
  
\end{document}